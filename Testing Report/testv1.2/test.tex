\documentclass[12pt, a4paper, oneside]{article}
\usepackage{amsmath, amsthm, amssymb, graphicx, float}
\title{SparseMatrixSolutor v1.2 Test Report}
\author{Anqi Lv\thanks{email:2200011316@stu.pku.edu.cn tel:18315391730}\\PHYS 2200011316}
\date{2025.10.6}
\renewcommand\thesection{\Roman{section}}
\renewcommand\thesubsection{\roman{subsection}}
\begin{document}
\maketitle
\section{Sparse Matrix}
\subsection{Construction and Print}
Try Matrix Construction:

SparseMatrix A(3, 3);

A.setValue(0, 0, 1.0);

A.setValue(0, 2, 2.0);

A.setValue(1, 1, 3.0);

A.setValue(2, 2, 5.0);

A.setValue(2, 0, 4.0);

Print A both in Dense and Sparse form. Output is correct.

\begin{figure}[H]
    \centering
    \includegraphics[width=5cm]{test1.png}
    \caption{Print}
\end{figure}

\subsection{Transpose}
Try transposing the Matrix A(defined above):

\begin{figure}[H]
    \centering
    \includegraphics[width=5cm]{test2.png}
    \caption{Transpose}
\end{figure}

Output is correct.

\subsection{Cut}
Try cutting Matrix into diagonal part, upper triangle part and lower triangle part.

\begin{figure}[H]
    \centering
    \includegraphics[width=5cm]{test4.png}
    \caption{Cut}
\end{figure}

Operation is correct.

\subsection{Add and Multyply}
Try defining Matrix B, and testing + operator and * operator.

\begin{figure}[H]
    \centering
    \includegraphics[width=5cm]{test3.png}
    \caption{Add and Multiply}
\end{figure}

Operation is correct.

\section{Matrix Solver}
I have write two method for solving matrix:

Gauss-Sider Iteration is suitable for most cases,especially when the diagonal element is dominant.

Conjuction Gradient is based on Krylov subspace.
It can only be used when the matrix is positive defined and symmetric.

\subsection{Correction Test}
I test a 6*6 Matrix when the RHS is 12.0, 20.0, 30.0, 60.0, 20.0, 30.0, for each equation respectively:

\begin{figure}[H]
    \centering
    \includegraphics[width=10cm]{test5.png}
    \caption{Correction Test}
\end{figure}

And the result is correct for each method.

Noticing that Conjuction Gradient's Iteration is obviously less than Gauss-Sider,and it's residual is much less than the other one. Conjuction Gradient is recommanded for positive symmetric case.


\subsection{Time Test}
I test a 100000*100000 Matrix when the RHS is periodic. There are three elements in each row:

\begin{figure}[H]
    \centering
    \includegraphics[width=5cm]{test6.png}
    \caption{Time Test}
\end{figure}

Only takes 39ms for both transposing and Gauss-Sider Iteration.

\section{PDE Solver}
\subsection{Dirichlet BC in Rectangle area}
I solve Poisson equation as:
\[
    \nabla^2 u = 0
\]
with Boundary Condition:
\[
    u(x,y)=x^2-y^2 (x=0||x=1||y=0||y=1)
\]
I make grid with interval x,y is both 0.01,meaning that there are 100*100 nodes.

The problem is solved rapidly:
\begin{figure}[H]
    \centering
    \includegraphics[width=10cm]{test7.png}
    \caption{Solve Poisson EQ}
\end{figure}

\subsection{Visualize}
In this part I choose Python language to Plot the result. Mention that the solution of the equation is obvious $u(x,y)=x^2-y^2$, the computing result is obviously right.

\begin{figure}[H]
    \centering
    \includegraphics[width=10cm]{test8.png}
    \caption{Visualize the Solution}
\end{figure}
\end{document}
